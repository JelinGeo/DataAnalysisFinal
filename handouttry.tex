\documentclass[
  man,
  floatsintext,
  longtable,
  nolmodern,
  notxfonts,
  notimes,
  colorlinks=true,linkcolor=blue,citecolor=blue,urlcolor=blue]{apa7}

\usepackage{amsmath}
\usepackage{amssymb}



\usepackage[bidi=default]{babel}
\babelprovide[main,import]{american}


% get rid of language-specific shorthands (see #6817):
\let\LanguageShortHands\languageshorthands
\def\languageshorthands#1{}

\RequirePackage{longtable}
\RequirePackage{threeparttablex}

\makeatletter
\renewcommand{\paragraph}{\@startsection{paragraph}{4}{\parindent}%
	{0\baselineskip \@plus 0.2ex \@minus 0.2ex}%
	{-.5em}%
	{\normalfont\normalsize\bfseries\typesectitle}}

\renewcommand{\subparagraph}[1]{\@startsection{subparagraph}{5}{0.5em}%
	{0\baselineskip \@plus 0.2ex \@minus 0.2ex}%
	{-\z@\relax}%
	{\normalfont\normalsize\bfseries\itshape\hspace{\parindent}{#1}\textit{\addperi}}{\relax}}
\makeatother




\usepackage{longtable, booktabs, multirow, multicol, colortbl, hhline, caption, array, float, xpatch}
\usepackage{subcaption}
\renewcommand\thesubfigure{\Alph{subfigure}}
\setcounter{topnumber}{2}
\setcounter{bottomnumber}{2}
\setcounter{totalnumber}{4}
\renewcommand{\topfraction}{0.85}
\renewcommand{\bottomfraction}{0.85}
\renewcommand{\textfraction}{0.15}
\renewcommand{\floatpagefraction}{0.7}

\usepackage{tcolorbox}
\tcbuselibrary{listings,theorems, breakable, skins}
\usepackage{fontawesome5}

\definecolor{quarto-callout-color}{HTML}{909090}
\definecolor{quarto-callout-note-color}{HTML}{0758E5}
\definecolor{quarto-callout-important-color}{HTML}{CC1914}
\definecolor{quarto-callout-warning-color}{HTML}{EB9113}
\definecolor{quarto-callout-tip-color}{HTML}{00A047}
\definecolor{quarto-callout-caution-color}{HTML}{FC5300}
\definecolor{quarto-callout-color-frame}{HTML}{ACACAC}
\definecolor{quarto-callout-note-color-frame}{HTML}{4582EC}
\definecolor{quarto-callout-important-color-frame}{HTML}{D9534F}
\definecolor{quarto-callout-warning-color-frame}{HTML}{F0AD4E}
\definecolor{quarto-callout-tip-color-frame}{HTML}{02B875}
\definecolor{quarto-callout-caution-color-frame}{HTML}{FD7E14}

%\newlength\Oldarrayrulewidth
%\newlength\Oldtabcolsep


\usepackage{hyperref}




\providecommand{\tightlist}{%
  \setlength{\itemsep}{0pt}\setlength{\parskip}{0pt}}
\usepackage{longtable,booktabs,array}
\usepackage{calc} % for calculating minipage widths
% Correct order of tables after \paragraph or \subparagraph
\usepackage{etoolbox}
\makeatletter
\patchcmd\longtable{\par}{\if@noskipsec\mbox{}\fi\par}{}{}
\makeatother
% Allow footnotes in longtable head/foot
\IfFileExists{footnotehyper.sty}{\usepackage{footnotehyper}}{\usepackage{footnote}}
\makesavenoteenv{longtable}

\usepackage{graphicx}
\makeatletter
\newsavebox\pandoc@box
\newcommand*\pandocbounded[1]{% scales image to fit in text height/width
  \sbox\pandoc@box{#1}%
  \Gscale@div\@tempa{\textheight}{\dimexpr\ht\pandoc@box+\dp\pandoc@box\relax}%
  \Gscale@div\@tempb{\linewidth}{\wd\pandoc@box}%
  \ifdim\@tempb\p@<\@tempa\p@\let\@tempa\@tempb\fi% select the smaller of both
  \ifdim\@tempa\p@<\p@\scalebox{\@tempa}{\usebox\pandoc@box}%
  \else\usebox{\pandoc@box}%
  \fi%
}
% Set default figure placement to htbp
\def\fps@figure{htbp}
\makeatother


% definitions for citeproc citations
\NewDocumentCommand\citeproctext{}{}
\NewDocumentCommand\citeproc{mm}{%
  \begingroup\def\citeproctext{#2}\cite{#1}\endgroup}
\makeatletter
 % allow citations to break across lines
 \let\@cite@ofmt\@firstofone
 % avoid brackets around text for \cite:
 \def\@biblabel#1{}
 \def\@cite#1#2{{#1\if@tempswa , #2\fi}}
\makeatother
\newlength{\cslhangindent}
\setlength{\cslhangindent}{1.5em}
\newlength{\csllabelwidth}
\setlength{\csllabelwidth}{3em}
\newenvironment{CSLReferences}[2] % #1 hanging-indent, #2 entry-spacing
 {\begin{list}{}{%
  \setlength{\itemindent}{0pt}
  \setlength{\leftmargin}{0pt}
  \setlength{\parsep}{0pt}
  % turn on hanging indent if param 1 is 1
  \ifodd #1
   \setlength{\leftmargin}{\cslhangindent}
   \setlength{\itemindent}{-1\cslhangindent}
  \fi
  % set entry spacing
  \setlength{\itemsep}{#2\baselineskip}}}
 {\end{list}}
\usepackage{calc}
\newcommand{\CSLBlock}[1]{\hfill\break\parbox[t]{\linewidth}{\strut\ignorespaces#1\strut}}
\newcommand{\CSLLeftMargin}[1]{\parbox[t]{\csllabelwidth}{\strut#1\strut}}
\newcommand{\CSLRightInline}[1]{\parbox[t]{\linewidth - \csllabelwidth}{\strut#1\strut}}
\newcommand{\CSLIndent}[1]{\hspace{\cslhangindent}#1}





\usepackage{newtx}

\defaultfontfeatures{Scale=MatchLowercase}
\defaultfontfeatures[\rmfamily]{Ligatures=TeX,Scale=1}





\title{The Importance of Summary Statistics and Techniques for Creating
Them in R}


\shorttitle{Summary Statistics and Techniques in R}


\usepackage{etoolbox}






\author{Jelin George}



\affiliation{
{Hochschule Fresenius - University of Applied Science}}




\leftheader{George}



\abstract{This document presents a concise overview of summary
statistics and their importance in R. Summary statistics - such as mean,
median, standard deviation, and frequency counts - capture the key
features of a dataset, enabling quick exploration and interpretation. R
provides powerful functions and visualization tools to efficiently
compute and present these statistics, making them essential for
simplifying data, identifying patterns, and supporting informed analysis
and decision-making. }

\keywords{summary statistics, R programming, data analysis, data
visualisation, data summarisation}

\authornote{\par{\addORCIDlink{Jelin George}{0000-0000-0000-0001}} 
\par{ }
\par{   The authors have no conflicts of interest to disclose.    }
\par{Correspondence concerning this article should be addressed to Jelin
George, Email: \href{mailto:george.jelin@stud-hs.fresenius.de}{george.jelin@stud-hs.fresenius.de}}
}

\makeatletter
\let\endoldlt\endlongtable
\def\endlongtable{
\hline
\endoldlt
}
\makeatother

\urlstyle{same}



\makeatletter
\@ifpackageloaded{caption}{}{\usepackage{caption}}
\AtBeginDocument{%
\ifdefined\contentsname
  \renewcommand*\contentsname{Table of contents}
\else
  \newcommand\contentsname{Table of contents}
\fi
\ifdefined\listfigurename
  \renewcommand*\listfigurename{List of Figures}
\else
  \newcommand\listfigurename{List of Figures}
\fi
\ifdefined\listtablename
  \renewcommand*\listtablename{List of Tables}
\else
  \newcommand\listtablename{List of Tables}
\fi
\ifdefined\figurename
  \renewcommand*\figurename{Figure}
\else
  \newcommand\figurename{Figure}
\fi
\ifdefined\tablename
  \renewcommand*\tablename{Table}
\else
  \newcommand\tablename{Table}
\fi
}
\@ifpackageloaded{float}{}{\usepackage{float}}
\floatstyle{ruled}
\@ifundefined{c@chapter}{\newfloat{codelisting}{h}{lop}}{\newfloat{codelisting}{h}{lop}[chapter]}
\floatname{codelisting}{Listing}
\newcommand*\listoflistings{\listof{codelisting}{List of Listings}}
\makeatother
\makeatletter
\makeatother
\makeatletter
\@ifpackageloaded{caption}{}{\usepackage{caption}}
\@ifpackageloaded{subcaption}{}{\usepackage{subcaption}}
\makeatother

% From https://tex.stackexchange.com/a/645996/211326
%%% apa7 doesn't want to add appendix section titles in the toc
%%% let's make it do it
\makeatletter
\xpatchcmd{\appendix}
  {\par}
  {\addcontentsline{toc}{section}{\@currentlabelname}\par}
  {}{}
\makeatother

%% Disable longtable counter
%% https://tex.stackexchange.com/a/248395/211326

\usepackage{etoolbox}

\makeatletter
\patchcmd{\LT@caption}
  {\bgroup}
  {\bgroup\global\LTpatch@captiontrue}
  {}{}
\patchcmd{\longtable}
  {\par}
  {\par\global\LTpatch@captionfalse}
  {}{}
\apptocmd{\endlongtable}
  {\ifLTpatch@caption\else\addtocounter{table}{-1}\fi}
  {}{}
\newif\ifLTpatch@caption
\makeatother

\begin{document}

\maketitle

\hypertarget{toc}{}
\tableofcontents
\newpage
\section[Introduction]{The Importance of Summary Statistics and
Techniques for Creating Them in R}

\setcounter{secnumdepth}{-\maxdimen} % remove section numbering

\setlength\LTleft{0pt}


Summary statistics are numerical values that describe the main features
of a dataset, such as its center and spread. They simplify complex data
into easily interpretable numbers, offering quick insights into trends
and variability, and serve as a foundation for further analysis. These
statistics provide a snapshot of the data, facilitating initial
exploration, quality checks, and communication of results.

R offers a rich ecosystem of functions and packages - such as summary(),
dplyr::summarise(), and visualization tools like histograms and boxplots
- that streamline the computation and presentation of summary statistics
for both numeric and categorical data. Their importance lies in
simplifying large datasets, revealing patterns and outliers, and laying
the groundwork for deeper statistical analyses and informed
decision-making.

As the first and often most critical step in any analytical workflow,
summary statistics in R empower analysts and researchers to understand,
compare, and communicate data-driven insights with clarity and
precision.

Summary statistics can be typically divided into:

\begin{enumerate}
\def\labelenumi{\arabic{enumi}.}
\item
  \textbf{Descriptive statistics}: Summarize the main features of a
  dataset (e.g., mean, median, mode). \emph{This will be our focus
  here.}
\item
  \textbf{Inferential statistics}: Make predictions or inferences about
  a population based on a sample (not the focus here).
\end{enumerate}

I would like to highlight a book, \emph{Making sense of statistics: A
conceptual overview}, (\citeproc{ref-oh2023making}{Oh \& Pyrczak, 2023})
which offers a clear and accessible introduction to key statistical
concepts for beginners. The book focuses on building conceptual
understanding of both descriptive and inferential statistics, using
simple explanations, practical examples, and step-by-step guidance. It
is designed to help students from any discipline gain confidence in
applying statistics to research and interpreting data effectively.

\newpage

\section{Key Measures in Summary
Statistics}\label{key-measures-in-summary-statistics}

\begin{itemize}
\tightlist
\item
  \textbf{Measures of Central Tendency}: Central tendency measures
  indicate where most values in a dataset fall.
\item
  \textbf{Measures of Dispersion}: Dispersion measures describe the
  spread of data.
\item
  \textbf{Measures of Shape and Distribution}: Describe the overall
  pattern and characteristics of how data values are distributed within
  a dataset.
\item
  \textbf{Visualization tools}: Histogram and boxplot are tools to help
  understand distribution of data better.
\item
  \textbf{Frequency table}: It shows how often each value or category
  appears in a dataset, making it easy to spot common or rare values and
  summarize the data.
\item
  \textbf{Cross-tabulations} (contingency tables): It shows how two or
  more categorical variables are related by displaying the count of
  observations for each combination of categories.
\end{itemize}

Watch this
\href{https://www.youtube.com/watch?v=yoPGwvUzjgQ}{\textbf{tutorial
video}} on descriptive statistics in R to get you started. Additionally,
the YouTube videos listed here may be helpful for understanding the code
chunks (\citeproc{ref-walker2023gtsummary}{Walker, 2023})
(\citeproc{ref-dre2024gentle}{Videos, 2024}).

\vspace{1cm}

\section{Summarizing Data Frames}\label{summarizing-data-frames}

We can summarize entire datasets by calculating statistics like mean,
median, minimum, maximum, standard deviation, and counts for each
variable. This provides a clear overview of the data's structure and key
features.

Practical application: Apply summary statistics functions to the Star
Wars dataset. Use functions such as summary(), mean(), median(), sd(),
and others to explore key variables like height, mass, and age. This
will help you quickly understand the central tendencies, variability,
and distribution patterns within the Star Wars data frame. Watch this
YouTube video \href{https://www.youtube.com/watch?v=4vSfbz9YMa0}{Return
of the Star Wars dataset} to understand the details of the dataset.

Furthermore, read
\href{https://modernstatisticswithr.com/index.html}{Modern Statistics
with R} to understand essential tools and techniques in contemporary
statistical data analysis, using the R programming language. The book
features numerous examples and over 200 exercises with worked solutions.
The online version is freely available and regularly updated, with
downloadable datasets for hands-on learning.

\vspace{1cm}

\section{Limitations}\label{limitations}

The future of summary statistics is shaped by growing data complexity,
computational advances, and artificial intelligence. As noted in
(\citeproc{ref-oscgarden2023bootstrapping}{Garden, 2023}), summary
statistics are increasingly combined with methods like bootstrapping and
machine learning to handle complex data. AI is automating and
personalizing summaries (\citeproc{ref-datatas2025future}{Datatas,
2025}), while enhanced visual tools make data exploration more intuitive
(\citeproc{ref-potter2010visualizing}{Potter et al., 2010}).
(\citeproc{ref-fan2014challenges}{Fan et al., 2014}) highlight the need
for new frameworks to address big data challenges, and
(\citeproc{ref-gelman2024stat50}{Gelman \& Vehtari, 2024}) emphasize
ongoing methodological innovation.

In sum, summary statistics are evolving into dynamic, automated tools
essential for understanding large and complex datasets.

\newpage

\section{Conclusion}\label{conclusion}

Summary statistics are fundamental to any data analysis process, serving
as the essential first step in understanding and interpreting datasets.
In R, summary statistics provide a concise overview of data
distributions, central tendencies, and variability, enabling analysts to
quickly assess data quality, detect anomalies, and guide subsequent
analytical decisions. The flexibility and power of R-through core
functions like summary(), mean(), sd(), and packages such as dplyr, to
efficiently compute and customize statistical summaries for both
ungrouped and grouped data.

According to (\citeproc{ref-lane2013descriptive}{Lane, 2013}), using R's
tools to automate calculations of summary statistics-such as mean,
median, standard deviation, range, and percentiles-enables users to
efficiently produce both overall and group-wise summaries. This not only
streamlines exploratory data analysis but also establishes a strong
basis for advanced statistical modeling and hypothesis testing. As Lane
emphasizes, mastering summary statistics in R allows analysts to extract
meaningful insights, make better decisions, and clearly communicate
results across various fields of research and business analytics.

\newpage

\section{References}\label{references}

\phantomsection\label{refs}
\begin{CSLReferences}{1}{0}
\bibitem[\citeproctext]{ref-datatas2025future}
Datatas. (2025, April 14). \emph{The future of AI-generated data
summarization for large reports}.
\url{https://datatas.com/the-future-of-ai-generated-data-summarization-for-large-reports/}

\bibitem[\citeproctext]{ref-fan2014challenges}
Fan, J., Han, F., \& Liu, H. (2014). Challenges of big data analysis.
\emph{National Science Review}, \emph{1}(2), 293--314.
\url{https://doi.org/10.1093/nsr/nwt032}

\bibitem[\citeproctext]{ref-oscgarden2023bootstrapping}
Garden, O. (2023, November 27). \emph{The 8 most important statistical
ideas: Bootstrapping and simulation-based inference}.
\url{https://osc.garden/blog/bootstrapping-and-simulation-based-inference/}

\bibitem[\citeproctext]{ref-gelman2024stat50}
Gelman, A., \& Vehtari, A. (2024). \emph{What are the most important
statistical ideas of the past 50 years?}
\url{https://www.stat.columbia.edu/~gelman/research/unpublished/stat50.pdf}

\bibitem[\citeproctext]{ref-lane2013descriptive}
Lane, D. M. (2013). Descriptive statistics. In \emph{Introduction to
statistics}. Rice University.
\url{https://onlinestatbook.com/2/introduction/descriptive.html}

\bibitem[\citeproctext]{ref-oh2023making}
Oh, D. M., \& Pyrczak, F. (2023). \emph{Making sense of statistics: A
conceptual overview}. Routledge.

\bibitem[\citeproctext]{ref-potter2010visualizing}
Potter, K., Kniss, J., Riesenfeld, R., \& Johnson, C. R. (2010).
Visualizing summary statistics and uncertainty. \emph{Computer Graphics
Forum}, \emph{29}(3), 823--832.
\url{https://www.sci.utah.edu/~kpotter/publications/potter-2010-VSSU.pdf}

\bibitem[\citeproctext]{ref-dre2024gentle}
Videos, D. E. R. (2024). \emph{Gentle r \#4: Basic summary statistics in
r with r studio {[}video{]}}. YouTube.
\url{https://www.youtube.com/watch?v=8XFmPP93w_Y}

\bibitem[\citeproctext]{ref-walker2023gtsummary}
Walker, L. (2023). \emph{Easy summary tables in r with gtsummary
{[}video{]}}. YouTube. \url{https://www.youtube.com/watch?v=gohF7pp2XCg}

\end{CSLReferences}

\newpage

\section{Affadative}\label{affadative}

I hereby affirm that this submitted paper was authored unaided and
solely by me. Additionally, no other sources than those in the reference
list were used. Parts of this paper, including tables and figures, that
have been taken either verbatim or analogously from other works have in
each case been properly cited with regard to their origin and
authorship. This paper either in parts or in its entirety, be it in the
same or similar form, has not been submitted to any other examination
board and has not been published.

I acknowledge that the university may use plagiarism detection software
to check my thesis. I agree to cooperate with any investigation of
suspected plagiarism and to provide any additional information or
evidence requested by the university.

Checklist:

\begin{itemize}
\tightlist
\item[$\square$]
  The handout contains 3-5 pages of text.
\item[$\square$]
  The submission contains the Quarto file of the handout.
\item[$\square$]
  The submission contains the Quarto file of the presentation.
\item[$\square$]
  The submission contains the HTML file of the handout.
\item[$\square$]
  The submission contains the HTML file of the presentation.
\item[$\square$]
  The submission contains the PDF file of the handout.
\item[$\square$]
  The submission contains the PDF file of the presentation.
\item[$\boxtimes$]
  The title page of the presentation and the handout contain personal
  details (name, email, matriculation number).
\item[$\square$]
  The handout contains a abstract.
\item[$\square$]
  The presentation and the handout contain a bibliography, created using
  BibTeX with APA citation style.
\item[$\square$]
  Either the handout or the presentation contains R code that proof the
  expertise in coding.
\item[$\square$]
  The handout includes an introduction to guide the reader and a
  conclusion summarizing the work and discussing potential further
  investigations and readings, respectively.
\item[$\square$]
  All significant resources used in the report and R code development.
\item[$\square$]
  The filled out Affidavit.
\item[$\square$]
  A concise description of the successful use of Git and GitHub, as
  detailed here: \url{https://github.com/hubchev/make_a_pull_request}.
\item[$\square$]
  The link to the presentation and the handout published on GitHub.
\end{itemize}

Jelin George, 28May2025, Cologne






\end{document}
